\documentclass[11pt]{article}
    \title{%
        CropFactorCalculator\\
        \large En räknare för fotografer}
    \author{Nils Korsfeldt}
    \date{Februari 2021}

    % General document formatting
    \usepackage[english]{babel}
    \usepackage[margin=1in]{geometry}
    \usepackage[parfill]{parskip}
    \usepackage[utf8]{inputenc}

    \usepackage[usestackEOL]{stackengine}
    \usepackage{outlines}
    \usepackage{blindtext, color, soulutf8}
    \usepackage[usestackEOL]{stackengine}
    \usepackage[document]{ragged2e}
    \usepackage{microtype}

    % Related to math
    \usepackage{amsmath,amssymb,amsfonts,amsthm}

\begin{document}

\maketitle

{\raggedleft\vfill{%
    Nils Korsfeldt \\ 
    Gymnasiearbete 100 poäng \\
    Klass 17ts \\
    Teknikprogrammet \\
    Läsåret 2019/2020 \\
    Handledare: Bondhon Shahriar Alam
}\par
}

\clearpage

\begin{abstract}
\normalsize
Denna rapport redogör för framtagandet samt utvecklingen av ett verkyg för att
räkna ut beskärningsfaktor/förlängningsfaktor för fotografi på olika format.
Rapporten omfattar vad som ledde till utvecklingen av detta, hur den utfördes,
samt hur den skulle kunna appliceras i professionella situationer och
amatörsituationer.\par

Undersökningen har utförts i form av korta samtal med bekanta fotointresserade
och även icke fotointresserade som har fått verktyget och sammanhanget
förklarat för sig, och utifrån det har det tagits fram en uppfattning om
efterfrågan för verktyget.\par
\bigskip
\textuparrow \ \hl{skriv om till engelska i slutet?}

\end{abstract}

\clearpage

\renewcommand{\contentsname}{Innehållsförteckning}
\tableofcontents

\clearpage

\section{Inledning}
\sloppy
Om du någonsin har använt kameror med olika sensorformat eller filmformat så
har du troligtvis stött på problemet att konvertera brännvidder och bländare
mellan dessa, eller, exempelvis, att jämföra ett objektiv på en sensor med
samma objektiv på en annan sensor som är större eller mindre.\par
\fussy

I detta projekt siktar jag på att lösa detta problem med ett enkelt verktyg som
låter en jämföra sensorer och brännvidder som förklarat. Jag har valt detta för
att jag är insatt i fotografi och har haft just det här problemet flera gånger,
samt för att jag tycker att själva tekniken bakom fotografi och filmografi är
intressant.\par

\section{Syfte och frågeställningar}
\hl{Syftet:}\ Syftet med detta arbete är att utreda varför det finns ett behov
för ett nytt verktyg som räknar ut beskärningsfaktor/förlängningsfaktor, alltså
...

samt hur detta nya verktyg presterar i jämförelse med de som redan finns
"på marknaden". Med prestation menas hur funktionellt verktyget är samt hur
lätt/intuitivt det är att använda.

Frågeställningen är som följande:\par

\hl{Frågeställning:} 
\begin{enumerate}
    \item Hur skulle detta verktyg kunna appliceras i professionellt arbete/en
        professionell situation? Det vill säga, hur skulle en professionell
        fotograf eller fotostudio kunna använda detta verktyg i sitt
        arbetsflöde?
    \item Vad uppmanade utvecklingen av detta verktyg och hur skiljer verktyget
        sig från sina konkurrenter? Orsak bakom framtagandet, fördelar och
        nackdelar gentemot liknande verktyg.
\end{enumerate}

\section{Material och metod}
\sloppy
För att få svar på dessa frågor har det utförts intervjuer med vänner som också
är fotointresserade. Vännerna i fråga är Jacob Nilsson Lehmusjärvi och Daniel
Stridh. Båda har tagit examen i estet med fotografisk inriktning från
NTI-Gymnasiet Stockholm och de anses av författaren vara pålitliga källor med
väl grundade åsikter och uttalanden i ämnet.\par

De följande frågorna har ställts, och siktar på att ge perspektiv på
frågeställningen och underlätta med att besvara den.
\fussy

\begin{itemize}
    \item Fråga 1: Vilken roll spelar fotografi i ditt vardagsliv?
    \item Fråga 2: Har du stött på problemet i fråga?
    \item Fråga 3: Skulle du använda detta verktyg om det fanns?
\end{itemize}

\clearpage

\section{Undersökning och resultat}
\textbf{Vilken roll spelar fotografi i ditt vardagsliv?}\par

Jacob: ”Numera inte så mycket, men förr; jättemycket. Jag tar dock några bilder
per dag med mobilkameran och fotar ibland med en "riktig" kamera, oftast en
analog.”\par

Daniel: ”Vardagsliv är väl lite av en överdrift, men jag är med i
Melodifestivalklubben och fotar ibland för dem på olika evenemang eller
tillställningar. Numera är mitt fotograferande mest förlagt till evenemang men
foto är fortfarande för mig en viktig hobby.”\par

\textbf{Har du stött på problemet i fråga?}\par

Jacob: ”Nej, men det bygger på att jag inte brukar använda exempelvis spegellös
digitalkamera när jag fotar. Dock en tid funderade jag på att skaffa en sådan
just på grund av att man kan använda äldre objektiv på den, eftersom jag redan
har flera. På grund av ekonomiska faktorn köpte jag inte en.”\par

\sloppy
Daniel: ”Jag har faktiskt köpt ett 50-220mm objektiv till min digitalkamera, men
som egentligen passar ett annat bajonettfäste. Så när jag monterar objektivet
med en adapter så kommer jag att få en annorlunda brännvidd än vad objektivet
skulle ge på fästet det egentligen passar.”\par
\fussy

\textbf{Skulle du använda detta verktyg om det fanns?}\par

Jacob: ”Ja, i det fallet att jag skulle äga en spegellös digitalkamera så
skulle det här nog komma väl till hands för att räkna på hur det skulle fungera
med mina gamla objektiv.”\par

Daniel: ”När mitt objektiv kommer fram så kommer jag nog att använda det, ja.”
\par

\bigskip
\sloppy
Resultatet av undersökningen är att Jacob har ett tydligt fotointresse och var
ett tag inne på att köpa en spegellös digitalkamera som han skulle kunna sätta
sina äldre objektiv på. Han säger att om han hade gjort det så skulle verktyget
komma till användning.\par

Daniel har ett likvärdigt fotointresse och fotar ibland på evenemang som press.
Han har köpt ett nytt objektiv till sin digitalkamera som passar ett annat
fäste, och när han använder en adapter så kommer han att få en annan brännvidd.
Han säger att när hans objektiv kommer fram så kommer han nog[sic] att använda
verktyget. \par
\fussy

\section{Analys och diskussion}
\sloppy
Intervjuobjekten har ett vardagligt intresse för fotografi, både som hobby och
mer professionellt. De har därför varit relevanta att fråga för att de har
tidigare kunskap i området. Dessutom så har de, som sagt, tagit examen från
NTI-Gymnasiet Stockholm inom estet med fotografisk inriktning. \par

Eftersom att de har visat intresse för verktygen så anser jag att det finns en
rimlig anledning att utveckla det. De båda skulle använda verktyget för att
förenkla sitt arbete inom fotografi. Jacob skulle kunna använda det om han hade
köpt en digitalkamera som han kan använda sina gamla objektiv på. I det fallet
att han söka jobb som exempelvis porträttfotograf skulle allt detta gå bra ihop;
gamla objektiv används ofta av porträttfotografer för de speciella effekter som
de ger. Även Daniel, som faktiskt har köpt ett nytt objektiv, kommer att kunna
få nytta av verktyget när han ska använda objektivet på sin digitalkamera. Detta
kommer att vara relevant i den professionella situationen att han fotar på
evenemang för exempelvis Melodifestivalklubben.\par

Det som har uppmanat utvecklingen av detta verktyg är att jag själv har stött på
problemet som det löser; att använda ett objektiv som egentligen inte passar på
en kamera och försöka lista ut vilken brännvidd och bländare man kommer få.
Att göra detta manuellt med en miniräknare är ett grovgöra, och en räknare
underlättar massivt.\par

Det finns en "konkurrent" till mitt verktyg som jag känner till, och det är
\emph{mmCalc}.\footnote{mmcalc.com} Jag brukade använda \emph{mmCalc} tills
jag kände att den inte räckte för mina behov längre. Detta leder oss till den
största {(första?)} fördelen som verktyget jag själv utvecklat har, och det
är att det går att ändra referensformatet. På \emph{mmCalc} är
referensformatet låst på 35mm (36x24mm) vilket naturligtvis försvårar om man
ska använda sitt objektiv på något annat än en kamera med den sensorstorleken.
För att förenkla; det enda som kan räknas ut på \emph{mmCalc} är
35mm-motsvarigheten till ens objektiv. På mitt verktyg går det däremot att välja
vilket referensformat som helst, så om du exempelvis ska använda ett objektiv
för 35mm-sensorer på en kamera med en APS-sensor så går det utan problem att
räkna på. En annan, mindre fördel är att det steglöst går att steglöst
specificera bländartal. På \emph{mmCalc} finns det endast ett fast antal
bländartal att välja på, mellan f/0.7 och f/32.\par

En nackdel, dock, är att det inte går att välja telekonverter eller 
vidvinkelkonverter i mitt verktyg. Dessa innebär ett tillbehör som skruvas
fast mellan kameran och objektivet, som kan antingen förlänga (tele) eller
förkorta (vidvinkel) brännvidden. Detta är användbart om man har ett ett fast
objektiv men vill ha det mer/mindre inzoomat.\footnote{Poggers is not academic
language}\par

\bigskip
\fussy

\bigskip
\bigskip
\bigskip
\bigskip

\hl{Anteckningar, ignorera:}\par
\title{Problem}
\begin{outline}
    \1 Läsa CSV till json genom react
        \2 Blobs
    \1 API requests till backend
        \2 Lokal testing - browser requests till localhost bryter mot no-cors
        \2 Lösning: tog bort backend
    \1 eval() = farligt
\end{outline}
\bigskip
\title{Todo}
\begin{outline}
    \1 Utveckla undersökning, gör fler intervjuer
    \1 Utveckla frågorna i frågeställningen *
    \1 Utveckla metoder
    \1 Fixa inledningen *
\end{outline}

\end{document}
