\documentclass[11pt]{article}
    \title{%
        CropFactorCalculator\\
        \large En räknare för fotografer}
    \author{Nils Korsfeldt}
    \date{February 2021}

    % General document formatting
    \usepackage[english]{babel}
    \usepackage[margin=1in]{geometry}
    \usepackage[parfill]{parskip}
    \usepackage[utf8]{inputenc}
    \usepackage[usestackEOL]{stackengine}
    \usepackage{outlines}
    \usepackage{blindtext, color, soul}
    
    % Related to math
    \usepackage{amsmath,amssymb,amsfonts,amsthm}

    % Other stuff
%    \usepackage{blindtext}

\begin{document}

\maketitle

{\raggedleft\vfill{%
    Nils Korsfeldt \\ 
    Gymnasiearbete 100 poäng \\
    Klass 17ts \\
    Teknikprogrammet \\
    Läsåret 2019/2020 \\
    Handledare: Bondhon Shahriar Alam
}\par
}
\clearpage

\begin{abstract}
\normalsize
Denna rapport redogör för framtagandet samt utvecklingen av ett verkyg för att räkna ut beskärningsfaktor/förlängningsfaktor för fotografi på olika format. Rapporten omfattar vad som ledde till utvecklingen av detta, hur den utfördes, samt hur den skulle kunna appliceras i professionella situationer och amatörsituationer.\\ \\*
Undersökningen har utförts i form av korta samtal med bekanta fotointresserade och även icke fotointresserade som har fått verktyget och sammanhanget förklarat för sig, och utifrån det har det tagits fram en uppfattning om efterfrågan för verktyget. \par
\bigskip
\textuparrow \ \hl{skriv om till engelska i slutet?}

\end{abstract}

\clearpage

\renewcommand{\contentsname}{Innehållsförteckning}
\tableofcontents

\clearpage

\section{Inledning}
Om du någonsin har använt kameror med olika sensorformat eller filmformat så har du troligtvis stött på problemet att konvertera brännvidder och bländare mellan dessa, eller, exempelvis, att jämföra ett objektiv på en sensor med samma objektiv på en annan sensor som är större eller mindre. I detta projekt siktar jag på att lösa detta problem med ett enkelt verktyg som låter en jämföra sensorer och brännvidder som förklarat. Jag har valt detta för att jag är insatt i fotografi och har haft just det här problemet flera gånger. Jag tycker att det är ett intressant ämne för att det finns mycket matte och fysik bakom det, samt för att själva tekniken bakom fotografi och filmografi är intressant. \par

\section{Syfte och frågeställningar}
\hl{Syftet:}Syftet med detta arbete är att utreda hur ett verktyg för att beräkna beskärningsfaktor eller förlängningsfaktor skulle tas fram samt varför det finns ett behov för ett sådant verktyg. \par 
Frågeställningen är som följande: 

% \hl{Frågeställning:} \
\begin{enumerate}
    \item Hur skulle detta verktyg kunna appliceras i professionellt arbete/en professionell situation?
    \item Vad uppmanade utvecklingen av detta verktyg och hur skiljer verktyget sig från sina konkurrenter?
\end{enumerate}

\section{Material och metod}
För att få svar på dessa frågor har jag utfört intervjuer med vänner som också är fotointresserade. Vännerna i fråga är Jacob Nilsson Lehmusjärvi och Daniel Stridh. Båda gick tidigare estet med fotografisk inriktning på NTI-Gymnasiet Stockholm och jag anser dem vara pålitliga källor med väl grundade åsikter och uttalanden i ämnet. \par
\begin{itemize}
    \item Fråga 1: Vilken roll spelar fotografi i ditt vardagsliv?
    \item Fråga 2: Har du stött på problemet i fråga?
    \item Fråga 3: Skulle du använda detta verktyg om det fanns?
\end{itemize}

\section{Undersökning och resultat}
Jacob, fråga 1: ”Numera inte så mycket, men förr; jättemycket. Jag tar dock några bilder per dag med mobilkameran och fotar ibland med en "riktig" kamera, oftast en analog.” \par
Jacob, fråga 2: ”Nej, men det bygger på att jag inte brukar använda exempelvis spegellös digitalkamera när jag fotar. Dock en tid funderade jag på att skaffa en spegellös digitalkamera just på grund av att man kan använda äldre objektiv, eftersom jag redan har flera. På grund av ekonomiska faktorn köpte jag inte en spegellös digitalkamera.” \par
Jacob, fråga 3: ”Ja, i det fallet att jag skulle äga en spegellös digitalkamera så skulle det här nog komma väl till hands för att räkna på hur det skulle fungera med mina gamla objektiv.” \par
Daniel, fråga 1: ”Vardagsliv är väl lite av en överdrift, men jag är med i melodifestivalklubben och fotar ibland för dem på olika evenemang eller tillställningar. Numera är mitt fotograferande mest förlagt till evenemang men foto är fortfarande för mig en viktig hobby.” \par
Daniel, fråga 2: ”Jag har faktiskt köpt ett 50-220mm objektiv till min fullframe sonykamera, men som egentligen passar ett annat bajonettfäste. Så när jag monterar objektivet med en adapter så kommer jag att få en annorlunda brännvidd än vad objektivet skulle ge på fästet det egentligen passar.” \par
Daniel, fråga 3: ”När mitt objektiv kommer fram så kommer jag nog att använda det, ja.” \par
\bigskip
Sammanfattat så har Jacob ett tydligt fotointresse och var ett tag inne på att köpa en spegellös digitalkamera som han skulle kunna sätta sina äldre objektiv på. Han säger att om han hade gjort det så skulle verktyget komma till användning. Daniel har ett likvärdigt fotointresse och fotar ibland på evenemang som press. Han har köpt ett nytt objektiv till sin digitalkamera som passar ett annat fäste, och när han använder en adapter så kommer han att få en annan brännvidd. Han säger att när hans objektiv kommer fram så kommer han nog[sic] att använda verktyget. \par

\bigskip
\bigskip
\bigskip
\bigskip
\title{Problem}
\begin{outline}
    \1 Läsa CSV till json genom react
        \2 Blobs
    \1 API requests till backend
        \2 Lokal testing - browser requests till localhost bryter mot no-cors
        \2 Lösning: tog bort backend
    \1 eval() = farligt
\end{outline}

\end{document}
